\chapter{Sequence of maxima of frieze patterns}
In \S \ref{s:fp}, we introduced the notion of an (arithmetic) frieze pattern of width $n$, where $n$ is a positive integer.

In particular, we showed that 

1) Each (arithmetic) frieze pattern of width $n$ takes on finitely many values (Corollary \ref{cor:ImageFinite}). In 
other words, for each $n$ and each $f \in {\rm Frieze}(n)$, there is a well-defined positive integer
\[
    u_n(f) := \max ( f (i,m) : (i,m)  \in \{1,\ldots,n \}\times \mathbb{Z}).
\]

2) For each $n$ there are finitely many arithmetic frieze patterns of width $n$ (Lemma \ref{l:frieze-non-empty} and 
Proposition \ref{prop:frieze-finite}). In particular, for each $n$, there is a well-defined
positive integer 
\[
    u_n := \max (u_n(f) : f \in  {\rm Frieze}(n)).
\]

We are now able to formulate and prove the main theorem of this section.
\begin{theorem}
    \label{MainTheorem}
    For all $n \geq 1$, we have 
    \[
        u_n = F_{n+2},
    \]
    where $(F_{n})_{n \in \mathbb{N}}$ is the {\it Fibonacci sequence}, i.e. the sequence recursively defined by 
    $F_0 = 0, F_1 = 1$ and the recursive formula
    \[
        F_n = F_{n-1} + F_{n-2}.
    \]
\end{theorem}
We break down the proof into two lemmas. 

\begin{lemma}
    \label{l:unLB}
    For all $n \geq 1$, we have 
    \[
        u_n \geq F_{n+2}.
    \]
\end{lemma}
\begin{proof}
    We construct, for each $n$, a frieze pattern containing the value $F_{n+2}$. By Lemma \ref{l:testCriteria}, 
    it is sufficient to specify an $(n+2)$-tuple $(a_0, \ldots, a_{n+1})$ of positive integers such that $a_0 = a_{n+1} =1$ and 
    \begin{equation}\label{eq:divis}
        a_i \mid a_{i-1} + a_{i+1}, \qquad i \in \{1, \ldots, n\}.
    \end{equation}
    1) If $n$ is odd, one sees by a direct computation that the $(n+2)$-tuple 
    \[
        (f_2,F_4, F_6, \ldots, F_{n+1}, F_{n+2}, F_n, F_{n-2}, \ldots, F_5, F_3, F_1), 
    \]
    satisfies the conditions of \eqref{eq:divis}. 

    2) If $n$ is even, the $(n+2)$-tuple 
    \[
        (F_2, F_4, F_6, \ldots, F_n, F_{n+2}, F_{n+1}, F_{n-1}, \ldots, F_5, F_3,F_1),
    \]
    satisfies the conditions of \eqref{eq:divis}. 
\end{proof}

We now show the reverse inequality.
\begin{lemma}
    \label{l:unUB}
    For all $n \geq 1$, we have 
    \[
        u_n \leq F_{n+2}.
    \]
\end{lemma}