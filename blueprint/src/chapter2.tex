
\chapter{Pandean sequences, flutes and the Fibonacci sequence}
\begin{definition}
    \label{def:pandean}
   % \lean{pandean}
    A sequence $(a_k)_{k \in \mathbb{N}^*}$ of positive integers is called pandean if $a_1 = 1$ and 
    if, for every $k > 1$, we have
    \[
        a_k \mid a_{k-1} + a_{k+1}.
    \]
    Given a pandean sequence $(a_k)$, if there exists a positive integer $n$ such that $a_k = a_{k+n-1}$ for all $k\in \mathbb{N}$, 
    the tuple $(a_1, \ldots, a_n)$ is called a Pan flute, or simply a flute, of height $n$. The set of all flutes of a given height $n$ 
    is denoted Flute$(n)$.
\end{definition}
Note that in a flute of height $n$, the first and last entries are equal to $1$. It is clear that the constant sequence consisting 
entirely of ones is pandean, and such a pandean sequence gives rise to a flute of height $n$ for any $n$.
In other words, we have the following.
\begin{lemma}
    \label{l:kFluteNonEmpty}
   % \lean{nDiagNonEmpty}
    \uses{def:pandean}
    For any positive integer $n$, the set Flute$(n)$ is non-empty.
\end{lemma}
Recall that the Fibonacci sequence $(F_k)_{k \in \mathbb{N}}$ is defined by $F_0 = 0, F_1 = 1$ and the recursive formula
    \[
        F_k = F_{k-1} + F_{k-2}.
    \]


\begin{lemma}
    \label{l:FibFlute}
   % \lean{FibDiag}
    \uses{def:pandean}
    1) If $n$ is odd, the $n$-tuple 
    \[
        (F_2,F_4, F_6, \ldots, F_{n-1}, F_{n}, F_{n-2}, F_{n-4}, \ldots, F_5, F_3, F_1), 
    \]
    is a Pan flute of height $n$.

    2) If $n$ is even, the $n$-tuple 
    \[
        (F_2, F_4, F_6, \ldots, F_{n-2}, F_{n}, F_{n-1}, F_{n-3}, \ldots, F_5, F_3,F_1),
    \]
    is a Pan flute of height $n$. 
\end{lemma}
\begin{proof}
    These are a tedious but straightforward calculation.
\end{proof}

\begin{lemma}
    \uses{def:pandean}
    \label{lem:FluteReduction}
  %  \lean{nDiagReduction}
    In a flute $(a_1, \ldots, a_n)$, there exists an index $i \in \{2,\ldots, n-1\}$ 
    such that $a_i = a_{i-1} + a_{i+1}$.
\end{lemma}
\begin{proof}

\end{proof}

\begin{proposition}
    \uses{def:pandean,lem:FluteReduction}
    \label{prop:FluteBounded}
  %  \lean{nDiagBounded}
    Fix a positive integer $n$, and let $(a_1, \ldots, a_n) \in {\rm Flute}(n)$. 
    For any $i \in \{1,\ldots, n\}$, we have $a_i \leq F_n$. 
\end{proposition}
\begin{proof}
   By induction on $n$, and using Lemma \ref{lem:nDiagReduction}.
\end{proof}
