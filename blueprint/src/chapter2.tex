
\chapter{Pandean sequences, flutes and the Fibonacci sequence}
\begin{definition}
    \label{def:flute}
    \lean{flute}
    \leanok
    A sequence $(a_k)$, indexed by $\mathbb{N}^*$ and consisting of positive integers is called pandean if $a_1 = 1$ and 
    if, for every $k > 1$, we have
    \[
        a_k \mid a_{k-1} + a_{k+1}.
    \]
    Given a pandean sequence $(a_k)$, if there exists a positive integer $n$ such that $a_k = a_{k+n-1}$ for all $k\in \mathbb{N}$, 
    the tuple $(a_1, \ldots, a_n)$ is called a Pan flute, or simply a flute, of height $n$. The set of all flutes of a given height $n$ 
    is denoted Flute$(n)$.
\end{definition}
Note that in a flute of height $n$, the first and last entries are equal to $1$. 
\begin{lemma}
    \label{l:nFluteNonEmpty}
    \lean{fluteSetNonEmpty,csteFlute}
    \uses{def:flute}
    \leanok
    For any positive integer $n$, the set Flute$(n)$ is non-empty.
\end{lemma}
\begin{proof}
    \leanok
    It is clear that the constant sequence consisting 
    entirely of ones is pandean, and such a pandean sequence gives rise to a flute of height $n$ for any $n$.
\end{proof}
Recall that the Fibonacci sequence $(F_k)_{k \in \mathbb{N}}$ is defined by $F_0 = 0, F_1 = 1$ and the recursive formula
    \[
        F_k = F_{k-1} + F_{k-2}.
    \]


\begin{lemma}
    \label{l:FibFlute}
    \lean{fib_flute_even,fib_flute_odd}
    \uses{def:flute}
    \leanok
    1) If $n$ is odd, the $n$-tuple 
    \[
        (F_2,F_4, F_6, \ldots, F_{n-1}, F_{n}, F_{n-2}, F_{n-4}, \ldots, F_5, F_3, F_1), 
    \]
    is a Pan flute of height $n$.

    2) If $n$ is even, the $n$-tuple 
    \[
        (F_2, F_4, F_6, \ldots, F_{n-2}, F_{n}, F_{n-1}, F_{n-3}, \ldots, F_5, F_3,F_1),
    \]
    is a Pan flute of height $n$. 
\end{lemma}
\begin{proof}
    These are a tedious but straightforward calculation.
\end{proof}

\begin{lemma}
    \uses{def:flute}
    \label{lem:FluteReduction}
    \lean{FluteReduction}
    \leanok
    In a flute $(a_1, \ldots, a_n)$, one of the following two statements holds.
    
    1) $a_2 = 1$ or $a_{n-1} =1$.
    
    2) There exists an index $i \in \{2,\ldots, n-1\}$ such that $a_i = a_{i-1} + a_{i+1}$.
\end{lemma}
\begin{proof}
    Suppose that 1) does not hold. In particular, $a_2 - a_1 > 0$ and $a_n - a_{n-1} < 0$. We prove that statement 2) holds by 
    contradiction. Thus, assume that for all $i \in \{2,\ldots, n-1\}$, we have $a_i \neq a_{i-1} + a_{i+1}$. Since the $a_i$ are positive integers,
    we necessarily have $a_{i-1} + a_{i+1} \geq 2 a_i$ for all $i \in \{2,\ldots, n-1\}$. Thus, for a given $i$, we have 
    \[
        a_{i+1} - a_i = a_{i+1} + a_{i-1} - a_{i-1} - a_i \geq 2 a_i - a_{i-1} - a_i = a_i - a_{i-1}.
    \]
    Gathering these inequalities, we obtain
    \[
        a_n - a_{n-1} \geq a_{n-1} - a_{n-2} \geq \ldots \geq a_2 - a_1 > 0,
    \]
    which contradicts the fact that $a_n - a_{n-1} < 0$.
\end{proof}

\begin{proposition}
    \uses{def:flute,lem:FluteReduction}
    \label{prop:FluteBounded}
    \lean{FluteBounded}
    \leanok
    Fix a positive integer $n$, and let $(a_1, \ldots, a_n) \in {\rm Flute}(n)$. 
    For any $i \in \{1,\ldots, n\}$, we have $a_i \leq F_n$. 
\end{proposition}
\begin{proof}
   Consider the statement 
   \[
    P_n: \text{If }(a_1, \ldots, a_n) \in {\rm Flute}(n) \text{, then }a_i \leq F_n \text{ for all }i \in \{1,\ldots, n\}.
   \]
   The proposition claims that $P_n$ holds for all $n \in \mathbb{N}^*$. We prove this by induction on $n$. 
   The base case $n = 1$ is clear, since the only flute of height $1$ is $(1)$, and $F_1 = 1$. 
   Similarly, the case $n = 2$ is clear, since the only flute of height $2$ is $(1,1)$, and $F_2 = 1$.
    Now, assume that $P_k$ holds for all $k$ up to and including a fixed $n \in \mathbb{N}^*$. Let $(a_1, \ldots, a_{n+1}) \in {\rm Flute}(n+1)$.
    By Lemma \ref{lem:FluteReduction}, we have either $a_2 = 1$ or $a_n = 1$, or there exists an index $i \in \{2,\ldots, n\}$ such that
    $a_i = a_{i-1} + a_{i+1}$. Suppose that $a_2 = 1$. Then $(a_1, a_3, \ldots, a_{n+1}) \in {\rm Flute}(n)$, and so by the induction hypothesis, 
    $a_i \leq F_n$ for all $i \in \{1,\ldots, n\}$. Since $F_n \leq F_{n+1}$, we have $a_i \leq F_{n+1}$ for all $i \in \{1,\ldots, n\}$. 
    A similar argument applies if $a_n = 1$. Now, suppose that there exists an index $i \in \{2,\ldots, n\}$ such that 
    \begin{equation}\label{eq:ai}
        a_i = a_{i-1} + a_{i+1}.
    \end{equation} 
    We claim that $(a_1, a_2, \ldots , a_{i-1}, \widehat{a_i}, a_{i+1}, \ldots, a_{n+1}) \in {\rm Flute}(n)$, where $\widehat{a_i}$ means we omit $a_i$.
    Again by the induction hypothesis, we have that 
    \begin{equation}\label{eq:ajs}
        a_j \leq F_n, \qquad j \neq i.
    \end{equation} 
    It remains to show that $a_i \leq F_{n+1}$. 
    To see this, note that \eqref{eq:ai} and \eqref{eq:ajs} together imply it is sufficient to show that either $a_{i-1}$ or $a_{i+1}$ is bounded 
    above by $F_{n-1}$. But recall that $a_{i-1}$ and $a_{i+1}$ both belong to the flute 
    $(a_1, a_2, \ldots , a_{i-1}, \widehat{a_i}, a_{i+1}, \ldots, a_{n+1}) $ of height $n$. Thus, the conditions of Lemma \ref{lem:FluteReduction} apply
    to this flute, so that by a reduction argument identical to the one above, there exists a flute of height $n-1$ containing either $a_{i-1}$ or $a_{i+1}$ (or both !), 
    whereby at least one of the two is bounded above by $F_{n-1}$. This completes the induction step, and the proposition follows.
\end{proof}
