\chapter{Arithmetic frieze patterns}
\begin{definition}
\label{def:arith-fp}
\uses{def:closed-fp} 
    A frieze pattern of width $n$ is said to be \textit{arithmetic} if it takes values in $\mathbb{Z}_{>0}$. 
\end{definition}

We denote by Frieze$(n)$ the set of all arithmetic frieze patterns of width $n$. 

\begin{lemma}
    \label{l:horiztonal-knit}
    \uses{def:closed-fp}
    An $\mathbb{Q}_{>0}$-valued frieze pattern of width $n$ is completely determined by its
    zeroth diagonal. Moreover, any $n$-tuple of positive rationals placed on the zeroth diagonal 
    gives rise to a $\mathbb{Q}_{>0}$-valued frieze pattern.
\end{lemma}

\begin{lemma}
    \uses{def:arith-fp}
    \label{l:lower-bound-crit}
    A frieze pattern $f : \{1,\ldots, n\} \times \mathbb{Z} \to \mathbb{Q}$ is arithmetic if and only if, 
    for all $i \in \{1, \ldots , n\}$, $f(i,0) \in \mathbb{Z}_{>0}$ and $f(i,0)$ divides $f(i-1,0) + f(i+1,0)$.
\end{lemma}
\begin{proof}
    \uses{l:continuant, l:horiztonal-knit}
    Suppose that $f$ is arithmetic. Then each $f(i,0)$ is integral, and the claim follows from Lemma \ref{l:continuant}. 
    Suppose now that for all $i \in \{1, \ldots , n\}$, $f(i,0) \in \mathbb{Z}_{>0}$ and $f(i,0)$ divides $f(i-1,0) + f(i+1,0)$. By 
    Lemma \ref{l:continuant}, $f$ takes values in $\mathbb{Z}$, and by Lemma \ref{l:horiztonal-knit}, $f$ takes values in $\mathbb{Q}_{>0}$. 
\end{proof}

\begin{theorem}
\label{thm:poly-to-fp}
\uses{def:arith-fp}
    Fix $n \in \mathbb{N}$. There is a set-theoretic bijection from the set of triangulations of the regular $n+3$-gon in the plane
    and the set Frieze$(n)$. In particular we have
    \[
        |\text{Frieze}(n)| = C_{n+1},
    \]
    where $C_n = \frac{1}{n+1}\binom{2n}{n}$ is the $n^{\rm th}$ Catalan number.
\end{theorem}

The remainder of this section is dedicated to the proof of Theorem \ref{thm:poly-to-fp}. 

