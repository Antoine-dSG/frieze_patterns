\chapter{Sequence of maxima of frieze patterns}
In \S \ref{s:fp}, we introduced the notion of an (arithmetic) frieze pattern of height $n$, where $n$ is a positive integer.

In particular, we showed that each (arithmetic) frieze pattern of height $n$ takes on finitely many values (Corollary \ref{cor:imageFinite}). 

We now show that there are finitely many arithmetic frieze patterns of height $n$. 
\begin{lemma}
    \label{l:friezeNonEmpty}
    \lean{friezeNonEmpty}
    \uses{l:nDiagNonEmpty, prop:testCriteria}
    Fix a positive integer $n$. The set Frieze$(n)$ is non-empty.
\end{lemma}
\begin{proof}
    The proof of Lemma \ref{l:nDiagNonEmpty} showed that $(1,1,\ldots, 1)$ is an $n$-diagonal. 
    By Proposition \ref{prop:testCriteria}, $(FR_n)^{-1}(1,1,\ldots, 1)$ is an arithmetic frieze pattern of 
    height $n$.
\end{proof}
\begin{proposition}
    \uses{prop:testCriteria}
    \label{prop:friezeFinite}
    \lean{friezeFinite}
    Fix a positive integer $n$. The set Frieze$(n)$ is finite. 
\end{proposition}
\begin{proof}

\end{proof}

This leads to the following two definitions.
\begin{definition}
    \label{def:unf}
    \uses{cor:imageFinite}
For each $n \in \mathbb{N}^*$ and each $f \in {\rm Frieze}(n)$, there is a well-defined positive integer
\[
    u_n(f) := \max ( f (i,m) : (i,m)  \in \{1,\ldots,n \}\times \mathbb{Z}).
\]
\end{definition}

\begin{definition}
    \label{def:un}
    \uses{def:unf,prop:friezeFinite,l:friezeNonEmpty}
    For each $n$, there is a well-defined
    positive integer, called the {\it maximum value} among arithmetic frieze patterns of height $n$, which we write as
    \[
        u_n := \max (u_n(f) : f \in  {\rm Frieze}(n)).
    \]
\end{definition}




We are now able to formulate and prove the main theorem of this section.
\begin{theorem}
    \label{mainTheorem}
    \lean{mainTheorem}
    \uses{def:un, l:unLB, l:unUB}
    For all $n \geq 1$, we have 
    \[
        u_n = F_{n}.
    \]    
\end{theorem}
We break down the proof into two lemmas. 

\begin{lemma}
    \label{l:unLB}
    \lean{unLB}
    \uses{def:un,l:testCriteria, def:fib}
    For all $n \geq 1$, we have 
    \[
        u_n \geq F_{n}.
    \]
\end{lemma}
\begin{proof}
    We construct, for each $n$, a frieze pattern containing the value $F_{n+2}$. By Proposition \ref{prop:testCriteria}, 
    it is sufficient to specify an $n$-tuple $(a_1, \ldots, a_n)$ of positive integers such that $a_1 = a_n =1$ and 
    \begin{equation}\label{eq:divis}
        a_i \mid a_{i-1} + a_{i+1}, \qquad i \in \{2, \ldots, n-1\}.
    \end{equation}
    1) If $n$ is odd, one sees by a direct computation that the $n$-tuple 
    \[
        (F_2,F_4, F_6, \ldots, F_{n-1}, F_{n}, F_{n-2}, F_{n-4}, \ldots, F_5, F_3, F_1), 
    \]
    satisfies the conditions of \eqref{eq:divis}. 

    2) If $n$ is even, the $n$-tuple 
    \[
        (F_2, F_4, F_6, \ldots, F_{n-2}, F_{n}, F_{n-1}, F_{n-3}, \ldots, F_5, F_3,F_1),
    \]
    satisfies the conditions of \eqref{eq:divis}. 
\end{proof}




% We might need this in the future
\iffalse
\begin{proposition}
    \label{prop:testCriteria}
    \uses{def:arith_fp,def:nDiag,l:positivePatternCharact}
    \lean{bijFriezeToDiag}
    Let $n$ be a fixed positive integer. The map $R_n$ introduced in Lemma \ref{l:positivePatternCharact} restricts 
    to a set-theoretic bijection, which we denote $FR_n$, from ${\rm Frieze}(n)$ to ${\rm Diag}(n)$. 
\end{proposition}
\begin{proof}
    By definition, the image of $FR_n$ lies in $(\mathbb{Z}_{>0})^n$. By Lemma \ref{l:pattern_nContinuant}, we have 
    \[
        f (i+1,0) f (2,i) = f (i,0) + f (i+2, 0), \qquad \forall i \in \{0,\ldots, n-1\}.
    \]
    Since $f$ is arithmetic, $f (2,i) \in \mathbb{Z}$, and therefore the image of $FR_n$ does indeed lie in Diag$(n)$. 
    By Lemma \ref{l:positivePatternCharact} $FR_n$ is injective. It 
    remains to show that $FR_n$ is surjective. Consider $(a_1, \ldots, a_n) \in {\rm Diag}(n)$, and set $f = (R_n)^{-1}(a_1, \ldots, a_n)$ 
    (c.f Lemma \ref{l:positivePatternCharact}). By the definition of $R_n$, $f$ is positive and $\mathbb{Q}$-valued; 
    it remains to show that $f$ is integer-valued. We begin by showing that $f (2,m) \in \mathbb{Z}$ for all $m \in \mathbb{Z}$. 
    By definition, for each 
    $i \in \{0,\ldots, n-2\}$, there exists $c_i \in \mathbb{Z}$ such that
    \[
        f(i+1,0) * c_i = f (i+2,0) + f (i,0).    
    \]
    Using the first equation in Lemma \ref{l:pattern_nContinuant}, we deduce that $c_i = f(2,i)$ for $i =0, \ldots, n-2$. 
    Moreover, $ f (2,n-1) = f (n-1,0) \in \mathbb{Z}$ by assumption. Thus we have proved that 
    $f (2,m) \in \mathbb{Z}$ for $m = 0, \ldots, n-1$. To see that $f (2,n) \in \mathbb{Z}$, note from 
    Lemma \ref{l:pattern_nContinuant} that $f (2,n) = f (n-1,1)$ can be expressed as a {\it polynomial} with 
    integer coefficients in the variables $f (2,1), f (2,2),\ldots , f (2,n-2)$. The claim for all $m$ then follows from 
    Proposition \ref{prop:trsltInv}. 

    To see how this implies that $f (i,m) \in \mathbb{Z}$ for all $i \in \{2, \ldots, n\}$, it suffices to 
    see, again from Lemma \ref{l:pattern_nContinuant}, that every $f (i,m)$ can be expressed as a {\it polynomial} with 
    integer coefficients in the variables $f (2,m), f (2,m+1),\ldots , f (2,m+i-2)$.
\end{proof}
\fi 