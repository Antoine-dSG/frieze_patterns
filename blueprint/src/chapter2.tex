
\chapter{n-Diagonals}
\begin{definition}
    \label{def:nDiag}
    \lean{nDiag}
    Let $n$ be a positive integer. By an $n$-diagonal, we mean any $n$-tuple $(a_1,\ldots, a_n)$ of positive integers 
    such that $a_1 = a_n$ and 
    \[
        a_i \mid a_{i-1} + a_{i+1}, \qquad i \in \{2, \ldots, n-1\}.
    \]
    We denote by ${\rm Diag}(n)$ the set of $n$-diagonals.
\end{definition}

We prove a couple of key properties about $n$-diagonals. 
\begin{lemma}
    \label{l:nDiagNonEmpty}
    \lean{nDiagNonEmpty}
    \uses{def:nDiag}
    Fix a positive integer $n$. The set Diag$(n)$ is non-empty.
\end{lemma}
\begin{proof}
    The $n$-tuple $(1,1,\ldots, 1)$ consisting entirely of $1$s is clearly an $n$-diagonal.
\end{proof}

\begin{definition}
    \label{def:fib}
    The Fibonacci sequence $(F_n)_{n \in \mathbb{N}}$ is defined by $F_0 = 0, F_1 = 1$ and the recursive formula
    \[
        F_n = F_{n-1} + F_{n-2}.
    \]
\end{definition}

The following $n$-diagonals will play an important role for us.
\begin{lemma}
    \label{l:FibDiag}
    1) If $n$ is odd, the $n$-tuple 
    \[
        (F_2,F_4, F_6, \ldots, F_{n-1}, F_{n}, F_{n-2}, F_{n-4}, \ldots, F_5, F_3, F_1), 
    \]
    is an $n$-diagonal.

    2) If $n$ is even, the $n$-tuple 
    \[
        (F_2, F_4, F_6, \ldots, F_{n-2}, F_{n}, F_{n-1}, F_{n-3}, \ldots, F_5, F_3,F_1),
    \]
    is an $n$-diagonal. 
\end{lemma}
\begin{proof}
    These are a tedious but straightforward calculation.
\end{proof}

\begin{lemma}
    \uses{def:nDiag}
    \label{lem:nDiagReduction}
    In an $n$-diagonal $(a_1, \ldots, a_n)$, there exists an index $i \in \{2,\ldots, n-1\}$ 
    such that $a_i = a_{i-1} + a_{i+1}$.
\end{lemma}
\begin{proof}

\end{proof}

\begin{proposition}
    \uses{def:nDiag}
    \label{prop:nDiagFinite}
    \lean{nDiagFinite}
    Fix a positive integer $n$, and let $(a_1, \ldots, a_n) \in {\rm Diag}(n)$. 
    For any $i \in \{1,\ldots, n\}$, we have $a_i \leq F_n$. 
\end{proposition}
\begin{proof}
   By induction on $n$, and using Lemma \ref{lem:nDiagReduction}.
\end{proof}
