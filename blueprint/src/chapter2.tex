
\chapter{Pandean sequences, flutes and the Fibonacci sequence}
\begin{definition}
    \label{def:pandean}
   % \lean{pandean}
    A sequence $(a_n)_{n \in \mathbb{N}}$ of positive integers is called pandean if $a_0 = 1$ and 
    if, for every $n \geq 1$, we have
    \[
        a_n \mid a_{n-1} + a_{n+1}.
    \]
    Given a pandean sequence $(a_n)$, if there exists a positive integer $k$ such that $a_n = a_{n+k}$ for all $n\in \mathbb{N}$, 
    the tuple $(a_0, a_1, \ldots, a_k)$ is called a Pan flute, or simply a flute, of height $k$. The set of all flutes of a given height $k$ 
    is denoted Flute$(k)$.
\end{definition}
It is clear that the constant sequence consisting entirely of ones is pandean, and such a pandean sequence gives rise to a flutes of height $k$ for any $k$.
In other words, we have the following.
\begin{lemma}
    \label{l:kFluteNonEmpty}
   % \lean{nDiagNonEmpty}
    \uses{def:pandean}
    For any positive integer $k$, the set Flute$(k)$ is non-empty.
\end{lemma}

\begin{definition}
    \label{def:fib}
    The Fibonacci sequence $(F_n)_{n \in \mathbb{N}}$ is defined by $F_0 = 0, F_1 = 1$ and the recursive formula
    \[
        F_n = F_{n-1} + F_{n-2}.
    \]
\end{definition}

\begin{lemma}
    \label{l:FibFlute}
   % \lean{FibDiag}
    \uses{def:pandean}
    1) If $n$ is odd, the $n$-tuple 
    \[
        (F_2,F_4, F_6, \ldots, F_{n-1}, F_{n}, F_{n-2}, F_{n-4}, \ldots, F_5, F_3, F_1), 
    \]
    is a Pan flute of height $n-1$.

    2) If $n$ is even, the $n$-tuple 
    \[
        (F_2, F_4, F_6, \ldots, F_{n-2}, F_{n}, F_{n-1}, F_{n-3}, \ldots, F_5, F_3,F_1),
    \]
    is a Pan flute of height $n-1$. 
\end{lemma}
\begin{proof}
    These are a tedious but straightforward calculation.
\end{proof}

\begin{lemma}
    \uses{def:pandean}
    \label{lem:FluteReduction}
  %  \lean{nDiagReduction}
    In a flute $(a_0, \ldots, a_n)$, there exists an index $i \in \{1,\ldots, n-1\}$ 
    such that $a_i = a_{i-1} + a_{i+1}$.
\end{lemma}
\begin{proof}

\end{proof}

\begin{proposition}
    \uses{def:pandean,lem:FluteReduction}
    \label{prop:FluteBounded}
  %  \lean{nDiagBounded}
    Fix a positive integer $n$, and let $(a_0, \ldots, a_n) \in {\rm Flute}(n)$. 
    For any $i \in \{1,\ldots, n\}$, we have $a_i \leq F_{n+1}$. 
\end{proposition}
\begin{proof}
   By induction on $n$, and using Lemma \ref{lem:nDiagReduction}.
\end{proof}
