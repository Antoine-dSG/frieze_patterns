\chapter{Sequence of maxima of frieze patterns}
In \S \ref{s:fp}, we introduced the notion of an (arithmetic) frieze pattern of height $n$, where $n$ is a positive integer.

In particular, we showed that 

1) Each (arithmetic) frieze pattern of height $n$ takes on finitely many values (Corollary \ref{cor:imageFinite}). 

2) For each $n$ there are finitely many arithmetic frieze patterns of height $n$ (Lemma \ref{l:friezeNonEmpty} and 
Proposition \ref{prop:friezeFinite}).

This leads to the following two definitions.
\begin{definition}
    \label{def:unf}
    \uses{cor:imageFinite}
For each $n \in \mathbb{N}^*$ and each $f \in {\rm Frieze}(n)$, there is a well-defined positive integer
\[
    u_n(f) := \max ( f (i,m) : (i,m)  \in \{1,\ldots,n \}\times \mathbb{Z}).
\]
\end{definition}

\begin{definition}
    \label{def:un}
    \uses{def:unf,prop:friezeFinite,l:friezeNonEmpty}
    For each $n$, there is a well-defined
    positive integer, called the {\it maximum value} among arithmetic frieze patterns of height $n$, which we write as
    \[
        u_n := \max (u_n(f) : f \in  {\rm Frieze}(n)).
    \]
\end{definition}

\begin{definition}
    \label{def:fib}
    The Fibonacci sequence $(F_n)_{n \in \mathbb{N}}$ is defined by $F_0 = 0, F_1 = 1$ and the recursive formula
    \[
        F_n = F_{n-1} + F_{n-2}.
    \]
\end{definition}


We are now able to formulate and prove the main theorem of this section.
\begin{theorem}
    \label{mainTheorem}
    \lean{mainTheorem}
    \uses{def:un, l:unLB, l:unUB}
    For all $n \geq 1$, we have 
    \[
        u_n = F_{n}.
    \]    
\end{theorem}
We break down the proof into two lemmas. 

\begin{lemma}
    \label{l:unLB}
    \lean{unLB}
    \uses{def:un,l:testCriteria, def:fib}
    For all $n \geq 1$, we have 
    \[
        u_n \geq F_{n}.
    \]
\end{lemma}
\begin{proof}
    We construct, for each $n$, a frieze pattern containing the value $F_{n+2}$. By Proposition \ref{prop:testCriteria}, 
    it is sufficient to specify an $n$-tuple $(a_1, \ldots, a_n)$ of positive integers such that $a_1 = a_n =1$ and 
    \begin{equation}\label{eq:divis}
        a_i \mid a_{i-1} + a_{i+1}, \qquad i \in \{2, \ldots, n-1\}.
    \end{equation}
    1) If $n$ is odd, one sees by a direct computation that the $n$-tuple 
    \[
        (F_2,F_4, F_6, \ldots, F_{n-1}, F_{n}, F_{n-2}, F_{n-4}, \ldots, F_5, F_3, F_1), 
    \]
    satisfies the conditions of \eqref{eq:divis}. 

    2) If $n$ is even, the $n$-tuple 
    \[
        (F_2, F_4, F_6, \ldots, F_{n-2}, F_{n}, F_{n-1}, F_{n-3}, \ldots, F_5, F_3,F_1),
    \]
    satisfies the conditions of \eqref{eq:divis}. 
\end{proof}

We now turn to the reverse equality.
\begin{proposition}
    \label{l:unUB}
    \lean{unUB}
    \uses{def:un,l:testCriteria, def:fib}
    For all $n \geq 1$, we have 
    \[
        u_n \leq F_{n}.
    \]
\end{proposition}
\begin{proof}
    We need to show that
    \[
        P_n : u_n \leq F_{n}
    \]
    holds for all $n \in \mathbb{N}^*$. Clearly, it suffices to show that $P_1$, $P_2$ hold and that $P_n \wedge P_{n+1} \Rightarrow P_{n+2}$
    holds. 
    
    $P_1$: It is immediate that $u_1 = 1 = F_1$. 

    $P_2$: It is immediate that $u_2 = 1 = F_2$.  

    $P_n \wedge P_{n+1} \Rightarrow P_{n+2}$: To be added...
\end{proof}