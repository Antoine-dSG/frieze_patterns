% In this file you should put the actual content of the blueprint.
% It will be used both by the web and the print version.
% It should *not* include the \begin{document}
%
% If you want to split the blueprint content into several files then
% the current file can be a simple sequence of \input. Otherwise It
% can start with a \section or \chapter for instance.
\chapter{Basics on frieze patterns}\label{s:fp}
\section{Closed frieze patterns}\label{ss:closed-fp}

% Definition of a frieze pattern of width $n$
\begin{definition}
\label{def:closed-fp}
    Fix $n \in \mathbb{N}$. A map $f : \{1,2,\ldots , n\} \times \mathbb{Z} \longrightarrow \mathbb{R}$ is called
    a \textit{frieze pattern of width n} if, for all $(i,m) \in \{1,2,\ldots , n\} \times \mathbb{Z}$, we have
    \[
        f(i,m) f(i,m+1) = 1 + f(i+1,m) f(i-1, m+1),
    \]
    where by convention we set $f(0,m) = f(n+1,m) = 1$ for all $m \in \mathbb{Z}$. 
\end{definition}

Consider the map $\rho_n: \{1,2,\ldots , n\} \times \mathbb{Z} \longrightarrow \{1,2,\ldots , n\} \times \mathbb{Z}$ given by
\begin{equation}  
\label{def:glide}
    \rho_n(i,m) = (n+1-i, m+i+1).
\end{equation}
Note that $\rho_n^2 : (i,m) \mapsto (i,m+n+3)$, and therefore $\rho_n$ is bijective. The main result in this section is the 
so-called \textit{glide symmetry} of frieze patterns. 

\begin{proposition}
\label{prop:glide-symm}
\uses{def:closed-fp, def:glide} 
Every frieze pattern of width $n$ is $\rho_n$-invariant, i.e. satisfies 
\[
    f(\rho_n(i,m)) = f(i,m), \qquad \forall (i,m) \in \{1,2,\ldots , n\} \times \mathbb{Z}.
\]
\end{proposition}
The remainder of this section is dedicated to the proof of Proposition \ref{prop:glide-symm}. 



\section{Arithmetic frieze patterns}
\begin{definition}
\label{def:arith-fp}
\uses{def:closed-fp} 
    A frieze pattern of width $n$ is said to be \textit{arithmetic} if it takes values in $\mathbb{Z}_{>0}$. 
\end{definition}

We denote by Frieze$(n)$ the set of all arithmetic frieze patterns of width $n$. The main result in this section 
is Conway and Coxeter's enumeration of arithmetic frieze patterns. 

\begin{theorem}
\label{thm:poly-to-fp}
\uses{def:arith-fp}
    Fix $n \in \mathbb{N}$. There is a set-theoretic bijection from the set of triangulations of the regular $n+3$-gon in the plane
    and the set Frieze$(n)$. In particular we have
    \[
        |\text{Frieze}(n)| = C_{n+1},
    \]
    where $C_n = \frac{1}{n+1}\binom{2n}{n}$ is the $n^{\rm th}$ Catalan number.
\end{theorem}

The remainder of this section is dedicated to the proof of Theorem \ref{thm:poly-to-fp}. 

