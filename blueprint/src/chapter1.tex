\chapter{Field-valued patterns}\label{s:fp}
Throughout this document, we will study frieze patterns of finite {\it height}. This terminology (as compared to finite {\it width} or 
finite {\it order}) is unconventional but more convenient for formalisation. For us, the height of a frieze pattern corresponds 
to the number of rows, including the rows of ones but {\it excluding} the rows of zeros.
Throughout this section, we fix an arbitrary field $F$. 


\begin{definition}
    \label{def:pattern_n}
    \lean{pattern_n}
    \leanok
        Fix $n \in \mathbb{N}^*$. A map $f : \{0,1,\ldots , n,n+1\} \times \mathbb{Z} \longrightarrow F$ is called
        an \textit{$F$-valued pattern of height n} if, 
        
        1) for all $m \in \mathbb{Z}$, $ f(0,m) = f (n+1,m) = 0$, 
        
        2) for all $m \in \mathbb{Z}$, $ f(1,m) = f (n,m) = 1$, and 
        
        3) for all $(i,m) \in \{1,2,\ldots , n\} \times \mathbb{Z}$, we have
        \begin{equation}\label{eq:diamond}
            f(i,m) f(i,m+1) = 1 + f(i+1,m) f(i-1, m+1).
        \end{equation}
    \end{definition}

An $F$-valued pattern $f$ of height $n$ is said to be {\it nowhere zero} if $f(i,m) \neq 0$, for all $i \in \{1,\ldots , n\}$ and for all $m \in \mathbb{Z}$.
\begin{lemma}
    \lean{pattern_nContinuant1,pattern_nContinuant2}
    \label{l:pattern_nContinuant}
    \uses{def:pattern_n}
    \leanok
    Let $f$ be a nowhere-zero $F$-valued pattern of height $n$. For all $m$, we have 
    \begin{align*}
        f(i+2,m) &= f(2,m+i) f(i+1,m) - f(i,m), \qquad i \in \{0, \ldots, n-1\} \\
        f (i,m) &= f (n-1,m) f (i+1,m-1) - f(i+2,m-2), \qquad i \in \{0,n-1\}. 
    \end{align*}
\end{lemma}
\begin{proof}
    We begin by proving the first statement. That is, we prove
    \[
        P_i : \forall m \in \mathbb{Z},  f(i+2,m) = f(2,m+i+1) f(i+1,m) - f(i,m),
    \] 
    for $ i \in \{0, \ldots, n-1\}$. We do so by induction on $i$. 

    Base case $P_0$: We have that for all $m \in \mathbb{Z}$, $f(2,m) f(1,m) - f(0,m) = f (2,m+1)*1 - 0 = f (2,m)$. 

    Inductive hypothesis. Suppose that our claim holds for some  $i \in \{0,\ldots, n-2\}$ fixed. Then, 
    \begin{align*}
        f (i+3,m) f (i+1,m+1) & = f (i+2,m) f (i+2,m+1)-1 \\
                            &= f (i+2,m) (f(2,m+i+1) f(i+1,m+1) - f(i,m+1)) -1 \\
                            &= f (i+2,m) f (2,m+i+1) f (i+1,m+1) - (f (i+2,m)f (i,m+1) + 1) \\
                            &= f (i+2,m) f (2,m+i+1) f (i+1,m+1) - f(i+1,m) f (i+1,m+1).
    \end{align*}
    Since $f$ is nowhere-zero, we may divide both sides of the equation by $f (i+1,m+1)$ to obtain the desired equality.

    The second statement is proved almost identically. Namely, we prove
    \[
        Q_i: \forall m \in \mathbb{Z}, f (i,m) = f (n-1,m) f (i+1,m-1) - f(i+2,m-2),
    \]
    by induction on $i$, starting with $i = n-1$ and proving the inductive step $Q_i \Rightarrow Q_{i-1}$. 

    Base case $Q_{n-1}$: for all $m \in \mathbb{Z}$, $ f (n-1,m) f (n,m-1) - f(n+1,m-2) = f (n-1,m)*1 - 0 = f (n-1,m)$. 

    Inductive hypothesis. Suppose that $Q_{i+1}$ holds for some fixed $i \in \{0,\ldots, n-2\}$. Then, 
    \begin{align*}
        f (i,m) f (i+2,m-1) &= f (i+1,m-1) f (i+1,m) - 1\\
                            &= f (i+1,m-1) (f (i+2,m-1) f (n-1,m) - f(i+3,m-2)) -1 \\
                            &= f (i+1,m-1)f (n-1,m) f (i+2,m-1) - (f (i+1,m-1) f (i+3,m-2) + 1) \\
                            &= f (i+1,m-1)f (n-1,m) f (i+2,m-1) - f (i+2,m-2) f (i+2,m-1).
    \end{align*}
    Again since $f$ is nowhere-zero, dividing by  $f (i+2,m-1)$ on both sides we obtain $Q_i$. 
\end{proof}

\begin{proposition}
    \label{prop:trsltInv}
    \lean{trsltInv}
    \uses{def:pattern_n, l:pattern_nContinuant}
    \leanok
    Let $f$ be a nowhere-zero $F$-valued pattern of height $n$. Then, for all $m \in \mathbb{Z}$ and all $i \in \{0,\ldots, n+1\}$, we have
    \[
        f(i,m) = f(i,m+n+1).
    \]
\end{proposition}
\begin{proof}
    \leanok
    We prove a stronger statement, called the \textit{glide symmetry} of frieze patterns. First, consider 
    the map $\rho_n: \{0,1,\ldots , n+1\} \times \mathbb{Z} \longrightarrow \{0,1,\ldots , n+1\} \times \mathbb{Z}$ given by
    \begin{equation}  
    \label{def:glide}
        \rho_n(i,m) = (n+1-i, m+i).
    \end{equation}
    We show that every nowhere-zero $F$-valued pattern of height $n$ is $\rho_n$-invariant, i.e. satisfies 
    \[
        f(\rho_n(i,m)) = f(i,m), \qquad \forall (i,m) \in \{0,1,\ldots , n+1\} \times \mathbb{Z}.
    \]
    The proposition will then follow by observing that $\rho_n^2 : (i,m) \mapsto (i,m+n+1)$. Thus, consider the statement
    \[
        P_i: \forall m \in \mathbb{Z}, f (i,m) = f (n+1-i,m+i),
    \]
    where $i \in \{0, \ldots, n+1\}$. To prove that $P_i$ holds for all $i$, it is sufficient to prove that $P_0, P_1$ hold, 
    and that $P_i \wedge P_{i+1} \Rightarrow P_{i+2}$. 

    $P_0:$ for all $m \in \mathbb{Z}, f(0,m) = 0 = f(n+1,m)$. 

    $P_1:$ for all $m \in \mathbb{Z}, f(1,m) = 1 = f(n,m+1)$. 

    Now suppose we are given $i \in \{0,1,\ldots , n-1\}$ such that $P_i$ and $P_{i+1}$ hold. Then, for any fixed $m \in \mathbb{Z}$, 
    we have 
    \begin{align*}
        f(i+2,m) &= f(2,m+i) f(i+1,m) - f(i,m)\\
                    &= f (n-1,m+i+2) f (n-i,m+i+1) - f (n+1- i, m + i) \\
                    &= f (n-i-1,m + i + 2).
    \end{align*}
\end{proof}

\begin{corollary}
    \label{cor:imageFinite}
    \lean{imageFinite}
    \uses{prop:trsltInv}
    Let $f$ be a nowhere-zero $F$-valued pattern of height $n$. Then, ${\rm Im}(f) := \{f (i,m) : i \in \{1,\ldots, n\}, m \in \mathbb{Z}\}$ is
    a finite set. 
\end{corollary}
\begin{proof}
    Consider the finite set $\mathcal{D} = \{(i,m) : i \in \{1,\ldots, n\}, m \in \{0,\ldots, n\}\}$. By Proposition \ref{prop:trsltInv}, 
    \[
        {\rm Im}(f) = \{f (i,m) : (i,m) \in \mathcal{D}\},
    \]
    and the right-hand side is obviously finite. 
\end{proof}



\iffalse
\begin{lemma}
    \label{l:testEqualPattern}
    \lean{testEqualPattern}
    \uses{def:pattern_n}
    Let $f,f'$ be two nowhere-zero $F$-valued pattern of height $n$. If $f (i,0) = f' (i,0)$ for all $i \in \{1,\ldots, n\}$, then $f = f'$.
\end{lemma}
\begin{proof}
    It suffices to prove that $f (i,m) = f '(i,m)$ for all $m \in \mathbb{Z}$. This is easily proved, recursively on the set 
    $\{1,\ldots, n\} \times \mathbb{Z}$ with respect to the total order 
    \[
        (i,m) < (j,k) \qquad \text{ if and only if } m < k \text{ or } m=k \text{ and } i < j,
    \]
    by noticing that $f(i,m+1)$ in \eqref{eq:diamond} is completely determined by $f(i,m), f (i+1,m)$ and $f (i-1,m+1)$ when 
    $f$ is nowhere-zero.  
\end{proof}
\begin{lemma}
    \label{l:positivePatternCharact}
    \lean{positivePatternCharact}
    \uses{def:pattern_n,def:positivePattern, l:testEqualPattern}
    The map which associates to a positive $\mathbb{Q}$-valued pattern $f$ of height $n$ the $n$-tuple 
    \[
       R_n(f) :=  (1,f (2,0), \ldots, f(n-1,0),1),
    \]
    is a set-theoretic bijection from the set of positive $\mathbb{Q}$-valued patterns of height $n$ to 
    $(\mathbb{Q}_{>0})^{n-2}$. 
\end{lemma}
\begin{proof}
    Similar to that of Lemma \ref{l:testEqualPattern}.
\end{proof}
\fi