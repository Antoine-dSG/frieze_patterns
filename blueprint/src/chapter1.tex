\chapter{Background on frieze patterns}\label{s:fp}
\section{Field-valued patterns}
Throughout this section, we fix an arbitrary field $F$. 
\begin{definition}\label{def:pattern}
    \leanok
% We need to make use of the command \lean ...
    An {\it infinite $F$-valued pattern} is a map $f : \mathbb{N}\times\mathbb{N} \to F$ such that 
    
    i) $\forall m, f(0,m) = 1$, and 

    ii) $\forall i \geq 1, \forall m$, we have the ``diamond rule'' 
    \[
        f (i,m) f(i,m+1) = 1 + f(i+1,m) f(i-1,m+1).
    \]
\end{definition}
We say that an $F$-valued pattern $f$ is {\it nowhere zero} if $f(i,m) \neq 0$, for all $i,m$.  
\begin{lemma}
    \label{l:inftyContinuant}
    \leanok
    \uses{def:pattern}
    Let $f$ be a nowhere-zero $F$-valued pattern. For all $i \geq 2$ and for all $m$, we have 
    \[
        f(i,m) = f(1,m+i-1) f(i-1,m) - f(i-2,m).
    \]
\end{lemma}
\begin{proof}
    By induction on $i$. The claim for $i =2$ and arbitrary $m \in \mathbb{Z}$ follows by the diamond rule. Thus suppose that the 
    claim holds for all $j$ up
    to (but not including) a fixed $i$, and for all $m \in \mathbb{Z}$.   
\end{proof}

The following corollary is sometimes referred to as the ``vertical knitting lemma''.  
\begin{corollary}
    \uses{l:inftyContinuant}
    \label{l:horiztonal-knit}
    In a nowhere-zero $F$-valued pattern $f$, each element $f(i,m)$ is a polynomial with integer coefficients in the 
    entries $\ldots, f(1,-2), f(1,-1), f(1,0), f(1,1), \ldots$. 
\end{corollary}
\begin{proof}

\end{proof}

We now turn to bounded patterns. 
\begin{definition}
    \label{def:pattern_n}
    \leanok
        Fix $n \in \mathbb{N}$. A map $f : \{1,2,\ldots , n\} \times \mathbb{Z} \longrightarrow F$ is called
        an \textit{$F$-valued pattern of width n} if, for all $(i,m) \in \{1,2,\ldots , n\} \times \mathbb{Z}$, we have
        \[
            f(i,m) f(i,m+1) = 1 + f(i+1,m) f(i-1, m+1),
        \]
        where by convention we set $f(0,m) = f(n+1,m) = 1$ for all $m \in \mathbb{Z}$. 
    \end{definition}
Again, an $F$-valued pattern $f$ of width $n$ is {\it nowhere zero} if $f(i,m) \neq 0$, for all $i,m$.
\begin{lemma}
    \label{l:pattern_nContinuant}
    \uses{def:pattern_n}
    Let $f$ be a nowhere-zero $F$-valued pattern. For all $i \geq 2$ and for all $m$, we have 
    \[
        f(i,m) = f(1,m+i-1) f(i-1,m) - f(i-2,m).
    \]
\end{lemma}
\begin{proof}
\end{proof}

\begin{proposition}
    \label{prop:trslt-inv}
    \uses{def:pattern_n, l:pattern_nContinuant}
    Let $f$ be a nowhere-zero $F$-valued pattern of width $n$. Then, for all $i,m$, we have
    \[
        f(i,m) = f(i,m+n+3).
    \]
\end{proposition}
\begin{proof}

\end{proof}

Consider the map $\rho_n: \{1,2,\ldots , n\} \times \mathbb{Z} \longrightarrow \{1,2,\ldots , n\} \times \mathbb{Z}$ given by
\begin{equation}  
\label{def:glide}
    \rho_n(i,m) = (n+1-i, m+i+1).
\end{equation}
Note that $\rho_n^2 : (i,m) \mapsto (i,m+n+3)$, and therefore $\rho_n$ is bijective. 

The following is called the \textit{glide symmetry}. 

\begin{corollary}
\label{prop:glide-symm}
\uses{def:glide, prop:trslt-inv} 
Every nowhere-zero $F$-valued pattern of width $n$ is $\rho_n$-invariant, i.e. satisfies 
\[
    f(\rho_n(i,m)) = f(i,m), \qquad \forall (i,m) \in \{1,2,\ldots , n\} \times \mathbb{Z}.
\]
\end{corollary}
\begin{proof}
\end{proof}

\section{Frieze patterns}
Throughout this section, we fix an arbitrary (totally ordered?) field $F$. 

\section{Arithmetic frieze patterns}

\iffalse
\begin{definition}
    \label{def:arith-fp}
    \uses{def:closed-fp} 
        A frieze pattern of width $n$ is said to be \textit{arithmetic} if it takes values in $\mathbb{Z}_{>0}$. 
    \end{definition}
    
    We denote by Frieze$(n)$ the set of all arithmetic frieze patterns of width $n$. 
    
    \begin{lemma}
        \label{l:horiztonal-knit}
        \uses{def:closed-fp}
        An $\mathbb{Q}_{>0}$-valued frieze pattern of width $n$ is completely determined by its
        zeroth diagonal. Moreover, any $n$-tuple of positive rationals placed on the zeroth diagonal 
        gives rise to a $\mathbb{Q}_{>0}$-valued frieze pattern.
    \end{lemma}
    
    \begin{lemma}
        \uses{def:arith-fp}
        \label{l:lower-bound-crit}
        A frieze pattern $f : \{1,\ldots, n\} \times \mathbb{Z} \to \mathbb{Q}$ is arithmetic if and only if, 
        for all $i \in \{1, \ldots , n\}$, $f(i,0) \in \mathbb{Z}_{>0}$ and $f(i,0)$ divides $f(i-1,0) + f(i+1,0)$.
    \end{lemma}
    \begin{proof}
        \uses{l:continuant, l:horiztonal-knit}
        Suppose that $f$ is arithmetic. Then each $f(i,0)$ is integral, and the claim follows from Lemma \ref{l:continuant}. 
        Suppose now that for all $i \in \{1, \ldots , n\}$, $f(i,0) \in \mathbb{Z}_{>0}$ and $f(i,0)$ divides $f(i-1,0) + f(i+1,0)$. By 
        Lemma \ref{l:continuant}, $f$ takes values in $\mathbb{Z}$, and by Lemma \ref{l:horiztonal-knit}, $f$ takes values in $\mathbb{Q}_{>0}$. 
    \end{proof}

\fi
