\chapter{Background on frieze patterns}\label{s:fp}
\section{Field-valued patterns}
Throughout this section, we fix an arbitrary field $F$. 


\begin{definition}
    \label{def:pattern_n}
    \lean{pattern_n}
        Fix $n \in \mathbb{N}^*$. A map $f : \{1,2,\ldots , n\} \times \mathbb{Z} \longrightarrow F$ is called
        an \textit{$F$-valued pattern of width n} if, for all $(i,m) \in \{1,2,\ldots , n\} \times \mathbb{Z}$, we have
        \begin{equation}\label{eq:diamond}
            f(i,m) f(i,m+1) = 1 + f(i+1,m) f(i-1, m+1),
        \end{equation}
        where by convention we set $f(0,m) = f(n+1,m) = 1$ for all $m \in \mathbb{Z}$. 
    \end{definition}
    In a frieze pattern of width $n$, we may assume without loss of generality that \eqref{eq:diamond} also holds for 
    $i=0$ and $i = n+1$, by setting $f(-1,m) = f(n+2,m) = 0$ for all $m \in \mathbb{Z}$. 

An $F$-valued pattern $f$ of width $n$ is said to be {\it nowhere zero} if $f(i,m) \neq 0$, for all $i \in \{1,\ldots , n\}$ and for all $m \in \mathbb{Z}$.
\begin{lemma}
    \label{l:pattern_nContinuant}
    \uses{def:pattern_n}
    Let $f$ be a nowhere-zero $F$-valued pattern of width $n$. For all $m$, we have 
    \begin{align*}
        f(i,m) &= f(1,m+i-1) f(i-1,m) - f(i-2,m), \qquad i \in \{1, \ldots, n+2\} \\
        f (i,m) &= f (n,m) f (i+1,m-1) - f(i+2,m-2), \qquad i \in \{-1,n\}. 
    \end{align*}
\end{lemma}
\begin{proof}
    We begin by proving the first statement. That is, we prove
    \[
        P_i : \forall m \in \mathbb{Z},  f(i,m) = f(1,m+i-1) f(i-1,m) - f(i-2,m),
    \] 
    for $ i \in \{1, \ldots, n+2\}$. We do so by induction on $i$. 

    Base case $P_1$: We have that for all $m \in \mathbb{Z}$, $f(1-1,m) f(1,m+1-1) - f(1-2,m) = 1 * f (1,m) - 0 = f (1,m)$. 

    Inductive hypothesis. Suppose that our claim holds for some  $i \in \{1,\ldots, n+1\}$ fixed. Then, 
    \begin{align*}
        f (i+1,m) f (i-1,m+1) & = f (i,m) f (i,m+1)-1 \\
                            &= f (i,m) (f (i-1,m+1)f (1,m+i) - f (i-2,m+1)) -1 \\
                            &= f (i,m) f (1,m+(i+1)-1) f (i-1,m+1) - (f (i,m)f (i-2,m+1) + 1) \\
                            &= f (i,m) f (1,m+(i+1)-1) f (i-1,m+1) - f(i-1,m) f (i-1,m+1).
    \end{align*}
    Since $f$ is nowhere-zero, we may divide both sides of the equation by $f (i-1,m+1)$ to obtain the desired equality.

    The second statement is proved almost identically. Namely, we prove
    \[
        Q_i: \forall m \in \mathbb{Z}, f (i,m) = f (n,m) f (i+1,m-1) - f(i+2,m-2),
    \]
    by induction on $i$, starting with $i = n$ and proving the inductive step $Q_i \Rightarrow Q_{i-1}$. 

    Base case $Q_n$: for all $m \in \mathbb{Z}$, $f(n,m) = 1*f(n,m)-0 = f (n,m) f (n+1,m-1) - f(n+2,m-2)$. 

    Inductive hypothesis. Suppose that $Q_{i+1}$ holds for some fixed $i \in \{-1,\ldots, n-1\}$. Then, 
    \begin{align*}
        f (i,m) f (i+2,m-1) &= f (i+1,m-1) f (i+1,m) - 1\\
                            &= f (i+1,m-1) (f (i+2,m-1) f (n,m) - f(i+3,m-2)) -1 \\
                            &= f (i+1,m-1)f (n,m) f (i+2,m-1) - (f (i+1,m-1) f (i+3,m-2) + 1) \\
                            &= f (i+1,m-1)f (n,m) f (i+2,m-1) - f (i+2,m-2) f (i+2,m-1).
    \end{align*}
    Again since $f$ is nowhere-zero, dividing by  $f (i+2,m-1)$ on both sides we obtain $Q_i$. 
\end{proof}

\begin{proposition}
    \label{prop:trslt-inv}
    \uses{def:pattern_n, l:pattern_nContinuant}
    Let $f$ be a nowhere-zero $F$-valued pattern of width $n$. Then, for all $i,m$, we have
    \[
        f(i,m) = f(i,m+n+3).
    \]
\end{proposition}
\begin{proof}
    We prove a stronger statement, called the \textit{glide symmetry} of frieze patterns. First, consider 
    the map $\rho_n: \{0,1,\ldots , n+1\} \times \mathbb{Z} \longrightarrow \{0,1,\ldots , n+1\} \times \mathbb{Z}$ given by
    \begin{equation}  
    \label{def:glide}
        \rho_n(i,m) = (n+1-i, m+i+1).
    \end{equation}
    We show that every nowhere-zero $F$-valued pattern of width $n$ is $\rho_n$-invariant, i.e. satisfies 
    \[
        f(\rho_n(i,m)) = f(i,m), \qquad \forall (i,m) \in \{0,1,\ldots , n+1\} \times \mathbb{Z}.
    \]
    The proposition will then follow by observing that $\rho_n^2 : (i,m) \mapsto (i,m+n+3)$. Thus, consider the statement
    \[
        P_i: \forall m \in \mathbb{Z}, f (i,m) = f (n+1-i,m+i+1),
    \]
    where $i \in \{0, \ldots, n+1\}$. To prove that $P_i$ holds for all $i$, it is sufficient to prove that $P_0, P_1$ hold, 
    and that $P_i \wedge P_{i+1} \Rightarrow P_{i+2}$. 

    $P_0:$ for all $m \in \mathbb{Z}, f(0,m) = 1 = f(n+1,m+1)$. 

    $P_1:$ Fix an arbitrary $m \in \mathbb{Z}$. By the second equation in Lemma \ref{l:pattern_nContinuant}, we know that 
    $f (-1,m+2) = f (n,m+2) f (0,m+1) - f (1,m)$. By re-arranging, we get that $f (1,m) = f (n,m+2)$. 

    Now suppose we are given $i \in \{0,1,\ldots , n-1\}$ such that $P_i$ and $P_{i+1}$ hold. Then, for any fixed $m \in \mathbb{Z}$, 
    we have 
    \begin{align*}
        f (i+2,m) &= f (1,m+i+1)f (i+1,m) - f (i,m) \\
                    &= f (n,m+i+3) f (n-i,m+i+2) - f (n- i+1, m + i + 1) \\
                    &= f (n-i-1,m + i + 3).
    \end{align*}
\end{proof}

\begin{corollary}
    \label{cor:ImageFinite}
    Let $f$ be a nowhere-zero $F$-valued pattern of width $n$. Then, ${\rm Im}(f) := \{f (i,m) : i \in \{1,\ldots, n\}, m \in \mathbb{Z}\}$ is
    a finite set. 
\end{corollary}
\begin{proof}
    Consider the finite set $\mathcal{D} = \{(i,m) : i \in \{1,n\}, m \in \{0,\ldots, n+2\}\}$. By Proposition \ref{prop:trslt-inv}, 
    \[
        {\rm Im}(f) = \{f (i,m) : (i,m) \in \mathcal{D}\},
    \]
    and the right-hand side is obviously finite. 
\end{proof}

\section{Arithmetic frieze patterns}
Throughout this section, we consider the so-called arithmetic frieze patterns.

\begin{definition}
    \label{def:arith-fp}
    \uses{def:closed-fp} 
        A $\mathbb{Q}$-valued frieze pattern of width $n$ is said to be \textit{arithmetic} if it takes values in $\mathbb{Z}_{>0}$. 
        We denote by Frieze$(n)$ the set of arithmetic frieze patterns of width $n$. 
\end{definition}

The following lemma will be important in what follows. 
\begin{lemma}
    \label{l:testCriteria}
    Let $n$ be a fixed positive integer. 

    1) If $f$ is an arithmetic frieze pattern of width $n$, then
    \[
        f (i,0) \mid f (i-1,0) + f (i+1,0), \qquad i \in \{1, \ldots, n\}.
    \]

    2) If an $(n+2)$-tuple $(a_0, \ldots, a_{n+1})$ of positive integers satisfies $a_0 = a_{n+1} =1$ and 
    \[
        a_i \mid a_{i-1} + a_{i+1}, \qquad i \in \{1, \ldots, n\},
    \]
    then there exists a (necessarily unique) arithmetic frieze pattern of width $n$ such that 
    \[
        f(i,0) = a_i, \qquad i \in \{1, \ldots, n\}
    \]
\end{lemma}
\begin{proof}
    1) follows from the first equation in Lemma \ref{l:pattern_nContinuant}. It remains to show 2). 
\end{proof}

It is not clear from the definitions that, for a given positive integer $n$, the set Frieze$(n)$ is non empty.
\begin{lemma}
    \label{l:frieze-non-empty}
    Fix a positive integer $n$. The set Frieze$(n)$ is non-empty.
\end{lemma}
\begin{proof}
    According to Lemma \ref{l:testCriteria}, there exists an arithmetic frieze pattern of width $n$, denoted $f$, such 
    that $f (i,0) = 1$ for $i \in \{1,\ldots , n\}$. 
\end{proof}
\begin{proposition}
    \label{prop:frieze-finite}
    Fix a positive integer $n$. The set Frieze$(n)$ is finite. 
\end{proposition}
\begin{proof}

\end{proof}


\iffalse

    
    We denote by Frieze$(n)$ the set of all arithmetic frieze patterns of width $n$. 
    
    \begin{lemma}
        \label{l:horiztonal-knit}
        \uses{def:closed-fp}
        An $\mathbb{Q}_{>0}$-valued frieze pattern of width $n$ is completely determined by its
        zeroth diagonal. Moreover, any $n$-tuple of positive rationals placed on the zeroth diagonal 
        gives rise to a $\mathbb{Q}_{>0}$-valued frieze pattern.
    \end{lemma}
    
    \begin{lemma}
        \uses{def:arith-fp}
        \label{l:lower-bound-crit}
        A frieze pattern $f : \{1,\ldots, n\} \times \mathbb{Z} \to \mathbb{Q}$ is arithmetic if and only if, 
        for all $i \in \{1, \ldots , n\}$, $f(i,0) \in \mathbb{Z}_{>0}$ and $f(i,0)$ divides $f(i-1,0) + f(i+1,0)$.
    \end{lemma}
    \begin{proof}
        \uses{l:continuant, l:horiztonal-knit}
        Suppose that $f$ is arithmetic. Then each $f(i,0)$ is integral, and the claim follows from Lemma \ref{l:continuant}. 
        Suppose now that for all $i \in \{1, \ldots , n\}$, $f(i,0) \in \mathbb{Z}_{>0}$ and $f(i,0)$ divides $f(i-1,0) + f(i+1,0)$. By 
        Lemma \ref{l:continuant}, $f$ takes values in $\mathbb{Z}$, and by Lemma \ref{l:horiztonal-knit}, $f$ takes values in $\mathbb{Q}_{>0}$. 
    \end{proof}

\fi
