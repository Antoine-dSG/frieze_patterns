\chapter{Symmetries of frieze patterns}\label{s:fp}
\begin{definition}
\label{def:closed-fp}
\leanok
    Fix $n \in \mathbb{N}$. A map $f : \{1,2,\ldots , n\} \times \mathbb{Z} \longrightarrow \mathbb{Q}$ is called
    a \textit{frieze pattern of width n} if, for all $(i,m) \in \{1,2,\ldots , n\} \times \mathbb{Z}$, we have
    \[
        f(i,m) f(i,m+1) = 1 + f(i+1,m) f(i-1, m+1),
    \]
    where by convention we set $f(0,m) = f(n+1,m) = 1$ for all $m \in \mathbb{Z}$. 
\end{definition}
It is often convenient to also set $f(-1,m) = f(n+2,m) = 0$ for all $m \in \mathbb{Z}$. The following will be of constant use for us.

\begin{lemma}
    \label{l:continuant}
    \uses{def:closed-fp}
    For all $i \in \{1, \ldots , n+1\}$, we have 
    \[
        f(i,m) = f(1,m+i-1) f(i-1,m) - f(i-2,m).
    \]
\end{lemma}
\begin{proof}
    By induction on $i \in \{1, \ldots, n+1\}$. The claim is clear for $i =1$ and arbitrary $m \in \mathbb{Z}$. Thus suppose that the 
    claim holds for all $j$ up
    to (but not including) a fixed $i \in \{2, \ldots, n+1\}$, and for all $m \in \mathbb{Z}$. Then  
\end{proof}

The following corollary is sometimes referred to as the ``vertical knitting lemma''.  
\begin{corollary}
    \uses{l:continuant}
    \label{l:horiztonal-knit}
    In a frieze pattern $f$ of width $n$, the element $f(i,m)$ is a polynomial with integer coefficients in the 
    entries $f(1,m), f(1,m+1), \ldots, f(1,m+i)$. 
\end{corollary}
\begin{proof}

\end{proof}

Consider the map $\rho_n: \{1,2,\ldots , n\} \times \mathbb{Z} \longrightarrow \{1,2,\ldots , n\} \times \mathbb{Z}$ given by
\begin{equation}  
\label{def:glide}
    \rho_n(i,m) = (n+1-i, m+i+1).
\end{equation}
Note that $\rho_n^2 : (i,m) \mapsto (i,m+n+3)$, and therefore $\rho_n$ is bijective. The following lemma describes how 
each diamond is affected by $\rho_n$.

\begin{lemma}
    \uses{def:glide}
    \label{l:glide-diamond}
    The map $\rho_n$ sends diamonds to diamonds. More precisely, we have
    \[
        \begin{matrix}
            & (i,m+1) &\\
            (i+1,m) && (i+1,m+1) \\
                & (i+2,m)
        \end{matrix} \longmapsto 
        \begin{matrix}
            & \rho_n(i+2,m) &\\
            \rho_n(i+1,m) && \rho_n(i+1,m+1) \\
                & \rho_n(i,m+1)
        \end{matrix}
    \]
\end{lemma}
\begin{proof}
    It is a straightforward calculation.
\end{proof}

The following is called the 
\textit{glide symmetry} of frieze patterns. 

\begin{proposition}
\label{prop:glide-symm}
\uses{def:closed-fp, def:glide} 
Every frieze pattern of width $n$ is $\rho_n$-invariant, i.e. satisfies 
\[
    f(\rho_n(i,m)) = f(i,m), \qquad \forall (i,m) \in \{1,2,\ldots , n\} \times \mathbb{Z}.
\]
\end{proposition}
\begin{proof}
    \uses{l:glide-diamond, l:continuant, l:glide-diamond}

\end{proof}


